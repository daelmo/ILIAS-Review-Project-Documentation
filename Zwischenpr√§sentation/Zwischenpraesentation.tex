\documentclass{beamer}
\usetheme{Dresden}
\usepackage[utf8]{inputenc}

\title{Zwischenpräsentation}
\author{Team Ilias}
\date{\today}

\begin{document}
\maketitle
\frame{\tableofcontents[]}

\section{Teamaufteilung}
\begin{frame}
	\frametitle{Aufteilung des Teams}
	\begin{tabular}{|c|c|}\hline
		Teammitglied & Aufgabe \\\hline
		Josephine Rehak & Chefprogrammiererin\\\hline
		Richard Mörbitz & Assistent\\\hline
		Max Friedrich & Administrator\\\hline
		Peter Merseburger & Testverantwortlicher\\\hline
		Julius Felchow & Sekretär\\\hline
	\end{tabular}
\end{frame} 
 
\section{Aufgabe}
\begin{frame} %%Eine Folie
  \frametitle{Einführung zur Thematik} %%Folientitel
  	Ilias ist eine E-Learning Plattform in der E-Klausuren 			erstellt werden können. Ein Fragenpool kann erstellt und die Fragen in der Klausur genutzt werden.\\
    Bisher fehlt eine Reviewmöglichkeit dieser Fragen.
\end{frame}

\begin{frame} %%Eine Folie
  \frametitle{Musskriterien} %%Folientitel
    \begin{itemize}
    	\item Entwickeln von 2 Plugins zum ... 
    		\begin{itemize}
    			 \item Erstellen reviewbarer Fragen
    			 \item Erstellen von Reviews
    			 \item manuellen Zuordnen von Reviewer und						Autor
    			 \item Betrachten der Reviews
			\end{itemize}    			
    	\item Unterstützung der Sprache Deutsch und Englisch
    \end{itemize}
\end{frame}

\begin{frame} %%Eine Folie
  \frametitle{Kannkriterien} %%Folientitel
  \begin{itemize}
  	\item Itemkonstruktion
  	\item Blueprint
    \item Fragenhistory
    \item zufällige Reviewverteilung
  \end{itemize}
\end{frame}

\section{Probleme}
\begin{frame} %%Eine Folie
  \frametitle{Problembehandlung} %%Folientitel
    \begin{itemize}
    		\item Serverstörungen in der Fakultät 
    		\item Nutzung von GitHub nach Absprache mit ILIAS-Mitarbeiter
    \end{itemize}
\end{frame}

\section{Diagramme}
\begin{frame} %%Eine Folie
  \frametitle{Entwurfsklassendiagramm} %%Folientitel
  \begin{Definition} %%Definition
    Eine Definition
  \end{Definition}
\end{frame}

\begin{frame} %%Eine Folie
  \frametitle{Sequenzdiagramme} %%Folientitel
  \begin{Definition} %%Definition
    Eine Definition
  \end{Definition}
\end{frame}

\begin{frame} %%Eine Folie
  \frametitle{Zustandsdiagramme} %%Folientitel
  \begin{Definition} %%Definition
    Eine Definition
  \end{Definition}
\end{frame}

\section{Prototyp}
\begin{frame} %%Eine Folie
  \frametitle{Zustandsdiagramme} %%Folientitel
  \begin{Definition} %%Definition
    Eine Definition
  \end{Definition}
\end{frame}
\end{document}