\documentclass{beamer}
\usetheme{Dresden}
\usepackage[utf8]{inputenc}

\title{Zwischenpräsentation}
\author{Team Ilias}
\date{\today}

\begin{document}
\maketitle
\frame{\tableofcontents[]}

 
\section{Aufgabe}
\begin{frame} %%Eine Folie
  \frametitle{Einführung zur Thematik} %%Folientitel
  \begin{Überblick} %%Definition
    Ilias ist eine E-Learning Plattform
  \end{Überblick}
\end{frame}

\begin{frame} %%Eine Folie
  \frametitle{Musskriterien} %%Folientitel
  \begin{Definition} %%Definition
    Eine Definition
  \end{Definition}
\end{frame}

\begin{frame} %%Eine Folie
  \frametitle{Kannkriterien} %%Folientitel
  \begin{Definition} %%Definition
    Eine Definition
  \end{Definition}
\end{frame}

\section{Probleme}
\begin{frame} %%Eine Folie
  \frametitle{Problembehandlung} %%Folientitel
  \begin{Definition} %%Definition
    Eine Definition
  \end{Definition}
\end{frame}

\section{Diagramme}
\begin{frame} %%Eine Folie
  \frametitle{Entwurfsklassendiagramm} %%Folientitel
  \begin{Definition} %%Definition
    Eine Definition
  \end{Definition}
\end{frame}

\begin{frame} %%Eine Folie
  \frametitle{Sequenzdiagramme} %%Folientitel
  \begin{Definition} %%Definition
    Eine Definition
  \end{Definition}
\end{frame}

\begin{frame} %%Eine Folie
  \frametitle{Zustandsdiagramme} %%Folientitel
  \begin{Definition} %%Definition
    Eine Definition
  \end{Definition}
\end{frame}

\section{Prototyp}
\begin{frame} %%Eine Folie
  \frametitle{Zustandsdiagramme} %%Folientitel
  \begin{Definition} %%Definition
    Eine Definition
  \end{Definition}
\end{frame}
\end{document}